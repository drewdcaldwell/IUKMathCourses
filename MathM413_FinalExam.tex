\documentclass{article}
\usepackage{graphicx} % Required for inserting images
\usepackage{amsthm}   % Required for proof environment
\usepackage{amsmath}   % Required for proof environment
\usepackage{titlesec}
\usepackage{amssymb}
\titleformat*{\subsection}{\normalfont}
\titleformat*{\section}{\normalfont\bfseries}
\usepackage{hyperref}


\begin{document}

\begin{titlepage}
\scshape
\centering
\raisebox{-\baselineskip}{\rule{\textwidth}{1px}}
\vspace{.3cm}
\rule{\textwidth}{.5px}
\vspace{.3cm}
{\huge{{Math-M413 Final Exam}}}\par \vspace{0.3cm}
\rule{\textwidth}{2px}

\large{Drew Caldwell}\\
\large{Indiana University - Kokomo}\\
\vspace{1.3cm}
\vfill
\today
\end{titlepage}

\section*{1. Consider the set $A = \{\frac{1}{k} + \frac{1}{2^n} : k \in \mathbb{N}, n \in \mathbb{N}\}$. Answer the following questions about this set.}
\subsection*{(a.) What are the limit points of $A$?}

The only limit point of $A$ is $0$. As both $n,k \to \infty$ the elements in $A$ get arbitrarly close to $0$ but never reaches $0$.

\subsection*{(b.) Is $A$ closed? If not, provide a sequence which has a limit point not in $A$.}

$A$ is not closed. $A$ does not contain it's limit point $0$.

\subsection*{(c.) Is $A$ compact?}

For $A$ to be compact it has to be bounded and closed. We know that it is not closed so it cannot be compact.

\subsection*{(d.) Is $\bar{A}$, the closure of $A$ compact? Outline a proof or explain why not.}

The closure of $A$ is $\bar{A} = A \cup \{0\}$. The union of $0$ to the $A$ is enough to make the set closed. Looking at 
individual parts of the setup of the set we see that $1/k$ and $1/2^n$ are both strictly decreasing, each are bounded. $1/k$ is bounded 
by $1$ and the set $1/2^n$ is bounded by $1/2$. Thus both of them individually are bounded by $1$, implying that their sum is bounded, 
making the set $A$ thus bounded.

\section*{2. Prove that if $A,B$ are subsets of $\mathbb{R}$ and $f : A \to B$ is one-to-one, then either
$f(a) \sim B$ or that the cardinality of $f(A)$ is strictly less than the cardinality of $B$. Using improvised symbols, this can be abbreviated 
as $\#(A) \le \#(B)$.}

\begin{proof}
Let $f: A \to B$ be a one-to-one function. We WTS that either $f(A) = B$ or that $\#(A) \le \#(B)$.

Consider the case where $f(A) = B$ here the elements of $A$ are getting directly mapped to $B$. Thus, 
every element of $B$ corresponds to an element of $A$ through $f$. So, $A$ and $B$ have the same number of elements. 
So we know that $\#(A) = \#(B)$.

Now if $f(A) \ne B$, then $A$ has less elements than $B$. Since $f$ is one-to-one each element of $A$ gets mapped to a 
distinct element in $B$. So, the set $f(A)$ is a strict subset of $B$. Making $\#f(A) < \#(B)$.

So, combining both cases we have show that either $f(A) = B$ or $\#(A) < \#(B)$. Making it so that,
$\#(A) \le \#(B)$.
\end{proof}

\section*{3. Show that $f(x) = x^5$ is continuous on $\mathbb{R}$ but not uniformly continuous on $\mathbb{R}$. You may use either the negation of the defintion or the fiven theorem.}
\subsection*{}

\section*{4. Provide an explicit example of a continuous function $f$ defined on an open set $A$ where $f(A)$ is closed. Be specific about $f$ and $A$.}

Let $f(x) = sin(x)$. $f$ is defined on the open interval $A = (0, 2\pi)$, and the image $f(A) = [-1,1]$, which is a closed set.

\section*{5. Where is the function below continuous?}
\subsection*{}

\section*{6. Let $h$ be continuous on the interval $A$ and $F$ be the set of points where $h$ is not one-to-one. That is,
 $F = \{f \in A : h(x) = h(y) \text{ for some } x \ne y \text{ and } y \in A\}$. }
\subsection*{}

\section*{7. Apply the following theorem}
\subsection*{Theorem 2. Let $f[a,b] \to \mathbb{R}$ be continuous. If $f(a)<L<f(b)$ or $f(a) > L > f(b)$. Then there exists a $c \in (a,b)$ such that $f(c) = L$.}
\subsection*{Let $g : [0,1] \to [0,1]$ be continuous. Prove that there is at least one $c \in [0,1]$ such that $g(c) = c$. Hint: consider the function $g(x) -x$.}

\section*{8. Prove that the sequence defined by $y_1 = 1$, $y_{n+1} = \frac{1}{4}(2y_n +3)$ has a limit.}

\section*{9. Decide if each of the following is true or false. Assume that $f:\mathbb{R} \to \mathbb{R}$ is continuous on $\mathbb{R}$.}
\subsection*{(a.) If $f(x) \le 0$ for all $x < 2$ then $f(2) \le 0$ as well.}
\subsection*{(b.) If $f(r) = 0$ for all $r \in \mathbb{Q}$, then $f(x) =0$ for all $x \in \mathbb{R}$.}
\subsection*{(c.) If $f(x_0) >0$ for some $x_0 \in \mathbb{R}$, then $f(x) >0$ for uncountably many points.}

\section*{10.}
\subsection*{(a.) Prove that $n^2 < n!$ for $n \ge 4$}
\begin{proof}

We will show that for $n \ge 4$ that $n^2 < n!$ by way of induction.

Basis:
\begin{align*}
    n^2 &= 4^2 = 16, \\
    n! &= 4! = 24, \\
    n^2 = 16 &< 24 = n!.
  \end{align*}

Inductive Step: Assume that the inequality holds for some integer \( n \ge 4 \), i.e.,
$n^2 < n!$.

We need to show that $(n+1)^2 < (n+1)!$. First, we expand $(n+1)^2:$

\begin{align*}
  (n+1)^2 &= n^2 + 2n + 1 \\
  n^2 &< n!  \text{   $($Using IH$)$}\\
  n^2 + 2n + 1 &< (n+1) \cdot n! \\
  (n+1)! &= (n+1) \cdot n!
\end{align*}

Thus, the inequality holds for $n+1$.


\end{proof}
\subsection*{(b.) Prove that the infintite series $\sum_{n=1}^{\infty} \frac{1}{n!}$ converges.}

\begin{proof}
    In order to show that the series $\sum_{n=1}^{\infty} \frac{1}{n!}$ converges we will apply the ratio test.

    Using the ratio test, the sequence is $(a_n) = 1/n!$. Then,

    \begin{align*}
        \frac{a_{n+1}}{a_n}
    \end{align*}
\end{proof}




\section*{11. Recall that for integers $a,b$, $a$ divides $b$ if there exists an integer $k$ such that $b = ak$. Prove that for all $n \in \mathbb{N}$, $3$ divides $n^3 -n$.}

\section*{12. Determine if each sequence below is Cauchy. You are not asked to find the limit of the sequence although this may be helpful.}
\subsection*{(a.) $(1/n)$}
\subsection*{(b.) $(1+(-1)^n)$}
\subsection*{(c.) $(\frac{n+1}{n})$}

\section*{13. Let $b$ be a positive real number. Prove that $\lim_{x \to 0} \frac{(x+1)^b - 1}{x} = b$}
\subsection*{(a.) Using the $\epsilon - \delta$ formulation of limit.}
\subsection*{(b.) Using the neighborhood function of limit.}

\end{document}

