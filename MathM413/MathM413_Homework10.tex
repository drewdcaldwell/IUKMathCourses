\documentclass{article}
\usepackage{graphicx} % Required for inserting images
\usepackage{amsthm}   % Required for proof environment
\usepackage{amsmath}   % Required for proof environment
\usepackage{titlesec}
\usepackage{amssymb}
\titleformat*{\subsection}{\normalfont}
\usepackage{hyperref}


\begin{document}

\begin{titlepage}
\scshape
\centering
\raisebox{-\baselineskip}{\rule{\textwidth}{1px}}
\vspace{.3cm}
\rule{\textwidth}{.5px}
\vspace{.3cm}
{\huge{{Math-M413 Homework 10}}}\par \vspace{0.3cm}
Chapter 3, Section 4: 1, 2, 6, 9
\rule{\textwidth}{2px}

\large{Drew Caldwell}\\
\large{Indiana University - Kokomo}\\
\vspace{1.3cm}
\vfill
\today
\end{titlepage}

\section*{Chapter 3, Section 4: 1, 2, 6, 9}

\section*{3.4.1}
If $P$ is a perfect set and $K$ is compact, is the intersection
 $P \cap K$ always compact? Always perfect?

\begin{proof}
By definition, a perfect set P is closed and has no isolated 
points. So, the intersection of a closed set P and a closed 
and bounded set K will give us a closed and bounded set $P 
\cap K$. So since $P \cap K$ is closed and bounded it is 
always compact. However $P \cap K$ is not always perfect. 
For example, let $P = \mathbb{R}$, then $P \cap K = K$ and 
we already know that $K$ is not necessarily perfect by our 
questions definition of $K$.
\end{proof}

\section*{3.4.2}
Does there exist a perfect set consisting of only rational 
numbers?
\newline
\newline
No. By Theorem 1.4.11 We know that $\mathbb{Q}$ is countable,
 and since $\mathbb{Q}$ is countable, any subset is also 
 countable. We also know that a non-empty perfect set is 
 uncountable. Since a set can't be countable and uncountable 
 we have a contradiction, and We know that there is no perfect 
 set that consists of only rational numbers.


\section*{3.4.6}
Prove Theorem 3.4.6
\subsubsection*{\textbf{Theorem 3.4.6}}
A set $E \subseteq R$ is connected if and only if, for all nonempty
disjoint sets $A$ and $B$ satisfying $E = A \cup B$, there always 
exists a convergent sequence $(x_n) \rightarrow x$ with $(x_n)$ 
contained in one of $A$ or $B$, and $x$ an element of the other.

\begin{proof}
Since this is an IFF proof we have to show that,
\begin{enumerate}
\item $E \subseteq \mathbb{R} \implies [(x_n) \rightarrow x  
\text{ where } (x_n),x \in A \text{ or } (x_n),x \in B]$

\begin{proof}
  Begin by assuming that $E \subseteq \mathbb{R}$ is connected. That is for any 
  nonempty disjoint sets $A$ and $B$ such that $E = A \bigcup B$, there exists a 
  convergent sequence $(x_n) \rightarrow x$ with $(x_n)$ in $A$ or $B$ and $x$ in 
  the set that $(x_n)$ is not in.
  
  Since $E$ is connected, it can't be separated into two disjoint nonempty open sets. If 
  $A \cap \bar{B} \ne \emptyset$ or $B \cap \bar{A} \ne \emptyset$, then we 
  would have a sequence $(x_n)$, WOLOG, contained in $A$ converging to a point $x \in \bar{B} 
  \implies x \in B$, which is expected.

  WOLOG, in the case of $\bar{A} \cap B \ne \emptyset$. Since $x \in \bar{A} \cap B$, $\exists (x_n) 
  \in A$ such tath $(x_n) \rightarrow x$. Thus, $x \in B$.
\end{proof}


\item $[(x_n) \rightarrow x \text{ where } (x_n),x \in A 
\text{ or } (x_n),x \in B] \implies  E \subseteq \mathbb{R}$
\begin{proof}

Now, assume that for every pair of non empty disjoint sets $A$ and $B$ such that 
$E = A \cup B$ there exists a sequence $(x_n) \subset A \text{or} (x_n) \subset B$ where 
$(x_n) \rightarrow x$ and $x \in A$ or $x \in B$, with $x$ being in set that $(x_n)$ is not in.

We WTS that $E$ is connected. Suppose BWOC that $E$ is disconnected. Thus, we can write $E$ as the 
union of two nonempty disjoint open sets $A$ and $B$, thus $E = A \cup B$ where $A \cap B = \emptyset$ 
and $A \ne \emptyset \ne B$.

WOLOG, assume that $\exists (x_n) \subset A \text{ such that } (x_n) \rightarrow x \text{ where } x \in B$. Since 
$A$ and $B$ are disjoint this contradicts are assumption that $A$ and $B$ are disjoint and open, because a sequence in $A$ 
that converges to an $x \in B$ forces a $x$ to belong to the closure of $A$, which implies that $A \cap \bar{B} \ne \emptyset$. 
This is a contradiction to what we assumed. 

Thus, $E$ must be connected.

\end{proof}
\end{enumerate}

  So, a set $E \subseteq R$ is connected if and only if, for all nonempty
  disjoint sets $A$ and $B$ satisfying $E = A \cup B$, there 
  always exists a convergent sequence $(x_n) \rightarrow x$ with 
  $(x_n)$ contained in one of $A$ or $B$, and $x$ an element of the other.
   

\end{proof}


\section*{3.4.9}
Let ${r_1, r_2, r_3, ...}$ be an enumeration of the rational 
numbers, and for each $n \in \mathbb{N}$ set $\epsilon_n =(1/2)^n$ . 
Define $O = \bigcup_{n=1}^{\infty} (V_{\epsilon_n}) (r_n), \text{ and let } F=O^c$



\subsection*{(a.) Argue that $F$ is closed, nonempty set consisting only of
 irrational numbers}

Let $\{r_1, r_2, r_3, ...\}$ be an enumeration of the rational numbers. Also let 
$(V_{\epsilon_n}) (r_n) = (r_n - \frac{1}{2^n}, r_n + \frac{1}{2^n})$ be epsilon 
neighborhoods around each rational number. If we were take the union of all of 
the neighborhoods we just defined, call it $O = \bigcup_{n=1}^{\infty} (V_{\epsilon_n}) (r_n)$, 
the complement of this union by definition would be all the elements not in any of the 
epsilon neighborhoods defined. 

Thus, $O$ is an open set by definition since it is the countable 
union of open intervals (example 3.2.2). Also, since $O$ is open, by Theorem 3.2.13, $F$ is 
closed. 

Since the irrational numbers are dense in $\mathbb{R}$, between any two numbers in 
$\mathbb{R}$, there exists an irrational number. So, since $O$ covers intervals surrounding 
every rational number, the irrational numbers outside of those intervals will be contained in $F$.
Thus $F$ contains irrational numbers, and is thus nonempty.

Thus, $F$ is a closed, nonempty set consisting of only irrational numbers.


\subsection*{(b.)Does $F$ contain any nonempty open intervals? is $F$ totally disconnected?}

An open interval $(a,b) \in F$ if every point in $(a,b)$ is not covered by any epsilon 
neighborhood $V_{\epsilon_n}(r_n)$. But, since the rational numbers are dense in $\mathbb{R}$, 
for any open interval around any irrational number, there are rational numbers that will fall within 
the interval $(a,b)$. Thus, $F$ does not contain any nonempty open intervals.

$F$ is totally disconnected if the only connected subsets are singletons. Since
 between any two irrational numbers, there exists rational numbers that are in $O$, any 
 segment of irratiional numbers does not form a continuous connection. Thus, $F$ is totally disconnected.

\subsection*{(c.) Is it possible to know whether $F$ is perfect? If not, can we modify this construction 
to produce a nonempty perfect set of irrational numbers?}

$F$ is perfect if it is closed and contains no isolated points. Although $F$ is closed, it contains isolated 
points. Consider an irrational number $x$. There exists a neighborhood around $x$ such that some rational 
numbers are included. Thus making the irrational points not limit points of $F$. So, since $F$ contains 
isolated points, it is not a perfect set.

If we were to modify it, we would need to create a nonempty perfect set of irrational numbers. We could do this by 
removing a countable dense set of rationals from a closed interval like $[0,1]$, in a specific way making sure that 
the limit points of irrational numbers are retained.











\section*{Notes}
\subsection*{\textbf{Definitions:}}
\subsubsection*{Perfect}
A set $P \subseteq \mathbb{R}$ is $\textbf{perfect}$ if it is closed and contains no isolated points.
\subsubsection*{Compact}
A set $K \subseteq \mathbb{R}$ is \textbf{compact} if every sequence in $K$ has a subsequence that converges to a limit that is also in $K$.

\subsection*{\textbf{Theorems:}}
\subsubsection*{Theorem 1.4.11}
(i) The set $\mathbb{Q}$ is countable. (ii) The set $\mathbb{R}$ is uncountable.

\end{document}

