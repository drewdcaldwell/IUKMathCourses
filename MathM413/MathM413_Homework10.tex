\documentclass{article}
\usepackage{graphicx} % Required for inserting images
\usepackage{amsthm}   % Required for proof environment
\usepackage{amsmath}   % Required for proof environment
\usepackage{titlesec}
\usepackage{amssymb}
\titleformat*{\subsection}{\normalfont}
\usepackage{hyperref}


\begin{document}

\begin{titlepage}
\scshape
\centering
\raisebox{-\baselineskip}{\rule{\textwidth}{1px}}
\vspace{.3cm}
\rule{\textwidth}{.5px}
\vspace{.3cm}
{\huge{{Math-M413 Homework 10}}}\par \vspace{0.3cm}
Chapter 3, Section 4: 1, 2, 6, 9
\rule{\textwidth}{2px}

\large{Drew Caldwell}\\
\large{Indiana University - Kokomo}\\
\vspace{1.3cm}
\vfill
\today
\end{titlepage}

\section*{Chapter 3, Section 4: 1, 2, 6, 9}

\section*{3.4.1}
If $P$ is a perfect set and $K$ is compact, is the intersection $P \cap K$ always compact? Always perfect?

\begin{proof}
By definition, a perfect set P is closed and has no isolated points. So, the intersection of a closed set P and a closed and bounded set K will give us a closed and bounded set $P \cap K$. So since $P \cap K$ is closed and bounded it is always compact. However $P \cap K$ is not always perfect. For example, let $P = \mathbb{R}$, then $P \cap K = K$ and we already know that $K$ is not necessarily perfect by our questions definition of $K$.
\end{proof}

\section*{3.4.2}
Does there exist a perfect set consisting of only rational numbers?
\newline
\newline
No. By Theorem 1.4.11 We know that $\mathbb{Q}$ is countable, and since $\mathbb{Q}$ is countable, any subset is also countable. We also know that a non-empty perfect set is uncountable. Since a set can't be countable and uncountable we have a contradiction, and We know that there is no perfect set that consists of only rational numbers.


\section*{3.4.6}
Prove Theorem 3.4.6
\subsubsection*{\textbf{Theorem 3.4.6}}
A set $E \subseteq R$ is connected if and only if, for all nonempty
disjoint sets $A$ and $B$ satisfying $E = A \cup B$, there always exists a convergent sequence $(x_n) \rightarrow x$ with $(x_n)$ contained in one of $A$ or $B$, and $x$ an element of the other.

\begin{proof}
Since this is an IFF proof we have to show that,
\begin{enumerate}
\item $E \subseteq \mathbb{R} \implies [(x_n) \rightarrow x  \text{ where } (x_n),x \in A \text{ or } (x_n),x \in B]$
\item $[(x_n) \rightarrow x \text{ where } (x_n),x \in A \text{ or } (x_n),x \in B] \implies  E \subseteq \mathbb{R}$
\end{enumerate}


First let it be true that a set $E \subseteq \mathbb{R}$ is connected. Since it is connected, we know that E is connected if and only if $E = A \cup B$ where $A$ and $B$ are disjoint and non empty subsets of $E$. Then it is either the case that $\bar{A} \cap B = \emptyset$ or $A \cap \bar{B} = \emptyset$. Since $A$ and $B$ are both disjoint sets, WOLOG, we'll look at the case of $\bar{A} \cap B = \emptyset$, since we would get the same logical answer if we chose $A \cap \bar{B} = \emptyset$. So consider an element $x \in \bar{A} \cap B$. Then,

\begin{equation} \label{eq1}
\begin{split}
 & x \in  \bar{A} \cap B\\
\implies & x \in \bar{A}  \text{ and }  x \in B
\end{split}
\end{equation}

So, since $x \in \bar{A}$. There exists a sequence $(x_n) \in A$, such that $(x_n) \rightarrow x$, since $\bar{A}$ is closed.
\newline
\newline
Now, to show the other way. Suppose that $E = A \cup B$, where $A$
and $B$ are nonempty disjoint subsets of $E$ and there is a sequence
$(x_n) \in A \text{ or } B$ such that $(x_n) \rightarrow x$ where and whatever
set $(x_n)$ is in, $x$ is in the other.
\newline
\newline
So, we WTS that $E$ is connected. To do so, we need to show that either
$\bar{A} \cap B \ne \emptyset$ or that $A \cap \bar{B} \ne \emptyset$.
 So, consider the sequence $(x_n)$. Let $(x_n) \in A$. WOLOG, we will
 show that $\bar{A} \cap B \ne \emptyset$. So, since $(x_n) \in A$, By 
 the definition of the question we know that $x \in B$. But, since 
 $(x_n) \rightarrow x$, this implies that $x \in \bar{A}$. Now, since
  $x \in \bar{A} \text{ and } x \in B$, $x \in \bar{A} \cap B \implies 
  \bar{A} \cap B \ne \emptyset$. Tus, $E$ is connected.

  So, a set $E \subseteq R$ is connected if and only if, for all nonempty
  disjoint sets $A$ and $B$ satisfying $E = A \cup B$, there always exists a convergent sequence $(x_n) \rightarrow x$ with $(x_n)$ contained in one of $A$ or $B$, and $x$ an element of the other.
   

\end{proof}


\section*{3.4.9}
Follow these steps to show that the Cantor set $ C_n = \bigcap\limits^\infty_{n=0} C_n$ described in Section 3.1 is totally disconnected in the sense described in Exercise 3.4.8.

\subsection*{(a.)  Given $x, y \in C$, with $x < y$, set $\epsilon = y - x$. For each $n = 0, 1, 2,... $, the set $C_n$ consists of a finite number of closed intervals. Explain why there must exist an $N$ large enough so that it is impossible for $x$ and $y$ both to belong to the same closed interval of $C_N$ .}



\subsection*{(b.) Argue that there exists a point$ z \notin C$ such that $x<z<y$. Explain how
this proves that there can be no interval of the form $(a, b)$ with $a<b$ contained
in C.}



\subsection*{(c.) Show that $C$ is totally disconnected.}










\section*{Notes}
\subsection*{\textbf{Definitions:}}
\subsubsection*{Perfect}
A set $P \subseteq \mathbb{R}$ is $\textbf{perfect}$ if it is closed and contains no isolated points.
\subsubsection*{Compact}
A set $K \subseteq \mathbb{R}$ is \textbf{compact} if every sequence in $K$ has a subsequence that converges to a limit that is also in $K$.

\subsection*{\textbf{Theorems:}}
\subsubsection*{Theorem 1.4.11}
(i) The set $\mathbb{Q}$ is countable. (ii) The set $\mathbb{R}$ is uncountable.

\end{document}

